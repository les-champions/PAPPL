%%% Exemple d'utilisation des macros PH sous TikZ %%%
\documentclass{article}

\usepackage[french]{babel}
\usepackage[utf8]{inputenc}
\usepackage[T1]{fontenc}

\usepackage{amsmath}  % Maths
\usepackage{amsfonts} % Maths
\usepackage{amssymb}  % Maths
%\usepackage{stmaryrd} % Maths (crochets doubles)

\usepackage{tikz}
\usetikzlibrary{arrows,shapes,shadows,scopes}
\usetikzlibrary{positioning}
\usetikzlibrary{matrix}
\usetikzlibrary{decorations.text}
\usetikzlibrary{decorations.pathmorphing}

\input{macros-ph}



\begin{document}

\begin{figure}[p]
\begin{center}

% Syntaxe des macros Process Hitting :
\begin{tikzpicture}
  
  % Définition d'une sorte :
  
  %\TSort{coordonnées}{nom}{nbr_processus}{orientation}
  %  avec orientation = l, r, t, b.
  \TSort{(0,0)}{a}{2}{l}
  \TSort{(3,0)}{b}{2}{r}
  
  % Définition d'une action :
  
  %\THit{a_i}{}{b_j}{.ancre1}{b_k}   % Partie frappe
  %\path[bounce, bend left/right] \TBounce{b_j}{}{b_k}{.ancre2};   % Partie bond
  %  avec a_i, b_j et b_k des processus qui existent,
  %  et ancre1, ancre2 des points cardinaux en anglais.
  %  On peut grouper les frappes pour gagner de la place.
  \THit{a_1}{}{b_0}{.west}{b_1}
  \path[bounce, bend left] \TBounce{b_0}{}{b_1}{.south west};
  
  % Définition d'un état :
  
  %\Tstate{état}
  %  où les processus de état sont séparés par des virgules.
  \TState{a_1,b_0}
  
\end{tikzpicture}

\end{center}
\caption{Exemple de schéma Process Hitting simple}
\end{figure}



\begin{figure}[p]
\begin{center}

\begin{tikzpicture}
  % Sortes
  \TSort{(0,0)}{a}{3}{r}
  \TSort{(-3,0.5)}{b}{2}{l}
  \TSort{(3,0.5)}{c}{2}{r}

  % Actions depuis b
  \THit{b_1}{}{a_0}{.west}{a_1}
  \THit{b_1}{}{a_1}{.north west}{a_2}
  \THit{b_0}{}{a_2}{.west}{a_1}
  \THit{b_0}{}{a_1}{.west}{a_0}
  \path[bounce, bend left=60]
    \TBounce{a_1}{}{a_2}{.south}
    \TBounce{a_0}{}{a_1}{.south}
  ;
  \path[bounce, bend right=60]
    \TBounce{a_2}{}{a_1}{.north}
    \TBounce{a_1}{}{a_0}{.north}
  ;

  % Actions depuis c
  \THit{c_1}{}{a_0}{.east}{a_1}
  \THit{c_1}{}{a_1}{.north east}{a_2}
  \THit{c_0}{}{a_2}{.east}{a_1}
  \THit{c_0}{}{a_1}{.east}{a_0}
  \path[bounce, bend right=60]
    \TBounce{a_1}{}{a_2}{.south east}
    \TBounce{a_0}{}{a_1}{.south east}
  ;
  \path[bounce, bend left=60]
    \TBounce{a_2}{}{a_1}{.north}
    \TBounce{a_1}{}{a_0}{.north east}
  ;
  % Auto-frappe
  \THit{c_0}{out=-40,in=40,selfhit}{c_0}{}{c_1}
  \path[bounce, bend right]
    \TBounce{c_0}{}{c_1}{.south east}
  ;

  % Actions depuis a
  \THit{a_2}{bend right}{b_1}{.north east}{b_0}
  \path[bounce, bend left=80]
    \TBounce{b_1}{out=100,in=140}{b_0}{.north}
  ;
\end{tikzpicture}

\end{center}
\caption{Exemple de schéma Process Hitting plus complexe}
\end{figure}

\end{document}

