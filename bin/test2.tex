\documentclass{article}
\usepackage[french]{babel}
\usepackage[utf8]{inputenc}
\usepackage[T1]{fontenc}
\usepackage{amsmath}  % Maths
\usepackage{amsfonts} % Maths
\usepackage{amssymb}  % Maths
\usepackage{tikz}
\usetikzlibrary{arrows,shapes,shadows,scopes}
\usetikzlibrary{positioning}
\usetikzlibrary{matrix}
\usetikzlibrary{decorations.text}
\usetikzlibrary{decorations.pathmorphing}
\input{macros-ph}  
\begin{document}
 % *** Style de TikZ 
\tikzstyle{style trait 10}=[black,thick]
\tikzstyle{style trait 11}=[red,thick]
\tikzstyle{style trait 12}=[green,thick]
\tikzstyle{style trait 13}=[blue,thick]

\tikzstyle{style fond 0}=[fill=black!30]
\tikzstyle{style fond 1}=[fill=red!30]
\tikzstyle{style fond 2}=[fill=green!30]
\tikzstyle{style fond 3}=[fill=blue!30]
\tikzstyle{style fond 5}=[draw=black,line width=1mm]

\begin{tikzpicture}
   \TSort{(0,0)}{TCRlig}{2}{l}
  \node[process, style fond 2] at (TCRlig_0.center) {}; 
\TState{TCRlig_0} 
  \THit{TCRlig_0}{out=-40, in=40, selfhit,}{TCRlig_0}{}{TCRlig_1} 
 \path[bounce, bend right] \TBounce{TCRlig_0}{}{TCRlig_1}{.south east}; 
\end{tikzpicture}
\end{document}